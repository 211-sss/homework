\documentclass{ctexart}
\usepackage{graphicx} % Required for inserting images
\usepackage{hyperref}
\title{命令行与Python}
\author{耿惠生 23020007028}
\date{September 2024}

\begin{document}

\maketitle

\section{练习内容}
\subsection{命令行环境}
\subsubsection{删除空目录}
rmdir
\subsubsection{移动、重命名}
mv
\subsubsection{合并}
cat
\subsection{Python入门基础}
\subsubsection{循环打印数字}
for i in range(1, 11):

    print(i)

结果:打印1到10的数字
\subsubsection{求和}
def add(a, b):

return a + b

result = add(3, 5)

print(result)

结果:输出3和5的和,即8
\subsubsection{读取文件内容}
with open("file.txt", "r") as file:

content = file.read()

print(content)

结果:打开file.txt,读取并打印文件内容
\subsubsection{利用模板}
import math

radius = 5

area = math.pi * radius **2

print(area)

结果:利用math,输出圆的面积
\subsubsection{斐波那契}
def fibonacci(n)

if n <= 1:

return n

else:

return fibonacci(n-1) + fibonacci(n-2)

n = 10

print("The", n, "th Fibonacci number is:", fibonacci(n))

结果:计算10的斐波那契数
\subsubsection{字典储存}
student = {

"name":"John",

"age":20,

"grade":"A"

}

print(student)

结果:输出学生信息
\subsubsection{阶乘}
def factorial(n):

    if n == 0:
    
        return 1
        
    else:
    
        return n * factorial(n-1)

num = 5

print("The factorial of", num, "is", factorial(num))

结果:输出5的阶乘,即120
\subsection{Python视觉设计}
\subsubsection{显示图像}
import cv2

image = cv2.imread('example.webp')

cv2.imshow('Example Image', image)

cv2.waitKey(0)

cv2.destroyAllWindows()

结果:利用cv2,读取example.webp,并显示该图片

\includegraphics[width=0.5\textwidth]{屏幕截图 2024-09-12 224356}
\subsubsection{保存图片}
cv2.imwrite('example.webp', image)

结果:保存图片
\subsubsection{灰化图像}
import cv2

image = cv2.imread('example.webp')

gray_image = cv2.cvtColor(image, cv2.COLOR_BGR2GRAY)

cv2.imshow('Grayscale Image', gray_image)

cv2.waitKey(0)

cv2.destroyAllWindows()

结果:彩色图片变为灰色

\includegraphics[width=0.5\textwidth]{屏幕截图 2024-09-12 224900.png}
\subsubsection{缩放图片}
import cv2

image = cv2.imread('example.webp')

scale_percent = 50 

width = int(image.shape[1] * scale_percent / 100)

height = int(image.shape[0] * scale_percent / 100)

dim = (width, height)

resized_image = cv2.resize(image, dim, interpolation=cv2.INTER_AREA)

cv2.imshow('Resized Image', resized_image)

cv2.waitKey(0)

cv2.destroyAllWindows()

结果:将图片缩小至20%

\includegraphics[width=0.5\textwidth]{屏幕截图 2024-09-12 225142}
\subsubsection{边缘检测}
import cv2

image = cv2.imread('example.webp', cv2.IMREAD_GRAYSCALE)

edges = cv2.Canny(image, 100, 200)

cv2.imshow('Edges', edges)

cv2.waitKey(0)

cv2.destroyAllWindows()

结果:对example.webp边缘检测,并输出检测结果

\includegraphics[width=0.5\textwidth]{屏幕截图 2024-09-12 225312}
\subsubsection{图形绘制与文本插入}
import cv2

image = cv2.imread('example.webp')

cv2.rectangle(image, (50, 50), (200, 200), (0, 255, 0), 2)

cv2.circle(image, (300, 300), 50, (0, 0, 255), -1)

cv2.putText(image, 'I love you', (50, 450), cv2.FONT_HERSHEY_SIMPLEX, 1, (255, 255, 255), 2)

cv2.imshow('Drawn Image', image)

cv2.waitKey(0)

cv2.destroyAllWindows()

结果:在图片上插入矩形,圆形,并输入文字“I love you”

\includegraphics[width=0.5\textwidth]{屏幕截图 2024-09-12 225504}
\subsubsection{高斯模糊}
blurred_image = cv2.GaussianBlur(gray_image, (5, 5), 0)

结果:将灰化图片模糊化

\includegraphics[width=0.5\textwidth]{屏幕截图 2024-09-12 225707}
\subsubsection{轮廓检测}
import cv2
import numpy as np

image = cv2.imread('example.jpg')

gray_image = cv2.cvtColor(image, cv2.COLOR_BGR2GRAY)

blurred_image = cv2.GaussianBlur(gray_image, (5, 5), 0)

edges = cv2.Canny(blurred_image, 50, 150)

contours, hierarchy = cv2.findContours(edges, cv2.RETR_EXTERNAL, cv2.CHAIN_APPROX_SIMPLE)

cv2.drawContours(image, contours, -1, (0, 255, 0), 2)

cv2.imshow('Image with Contours', image)

cv2.waitKey(0)

cv2.destroyAllWindows()

结果:绘制图片轮廓并显示轮廓检测后的图片

\includegraphics[width=0.5\textwidth]{屏幕截图 2024-09-12 225825}
\subsubsection{旋转图片}
from PIL import Image

def rotate_image(image_path, angle):

    image = Image.open(image_path)
    
    rotated_image = image.rotate(angle)
    
    rotated_image.save("rotated_image.jpg")

rotate_image("example.webp", 90)

结果:example.webp旋转了90度
\
\
\section{使用感悟}
\subsection{Python视觉设计 }
\subsubsection{简洁}
Python语法简洁,这使我们在视觉设计中实现复杂的算法和功能,使用起来十分便利。 
\subsubsection{库丰富}
Python有许多用于图像处理和视觉设计的库,如PIL、OpenCV、Matplotlib等。这些库提供了丰富的功能,是我们用简单的代码实现强大的功能。  
\section{github地址}
\url{https://github.com/211-sss/homework}
\end{document}
