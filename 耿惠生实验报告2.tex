\documentclass{ctexart}
\usepackage{graphicx} % Required for inserting images
\usepackage{hyperref}
\title{Shell与Vim}
\author{耿惠生 23020007028}
\date{September 2024}

\begin{document}

\maketitle

\section{练习内容}
\subsection{Shell}
\subsubsection{ls拓展操作}
ls -lah --sort=time --color  

-l:以长格式列出文件,显示详细信息

-a:包括所有文件和隐藏文件 

-h:以人类可读的格式显示文件大小 

--sort=time:根据最近访问的时间排序文件 

--color:以彩色文本显示输出结果 
\subsubsection{课后习题2}
vim:
\# marco.sh 

 MARCO\_DIR="" 

 
 marco() \{ 
 
 MARCO\_DIR=\$(pwd)
 
 echo "Current directory saved: \$MARCO\_DIR"
 
 \}
 
 polo() \{ 
 
 if [ -z "\$MARCO\_DIR" ]; then
 
 echo "No directory saved. Please run 'marco' first."
 
 else 
 
 cd "\$MARCO\_DIR" || echo "Failed to change directory to \$MARCO\_DIR"
 
 echo "Changed directory to: \$MARCO\_DIR" fi 
 
 \}  
\subsubsection{课后习题3}
\#!/bin/bash 

COMMAND="error.sh"


LOG\_FILE="output.txt"

RUN\_COUNT=0 

while true; do 

 \$COMMAND >> "\$LOG\_FILE" 2>\&1 
 
 if [ \$? -ne 0 ]; then
 
 echo "Command failed after \$RUN\_COUNT runs." break
 
 fi 

 RUN\_COUNT=\$((RUN\_COUNT + 1)) 
 
 done 

 echo "Output and errors from the command:" 
 
 cat "\$LOG\_FILE"  
\subsubsection{课后习题4}
find
/home/learn/shell
-type
f 
-name "*.html" 
-print0 
| 
zip
-@
output.zip

-type f : 只查找文件 

\verb|-name "*.html"|: 查找所有扩展名为 \verb|.html| 的文件 

\verb|-print0|: 以 null 字符分隔输出文件名,这样可以正确处理文件名中的空格和特殊字符 

\verb|zip -@ output.zip|: 从标准输入读取文件名并将它们压缩到 \verb|output.zip| 文件中 
\subsubsection{课后习题5}
find
/home/learn/shell 
-type
f
-atime
-7 

\verb|-atime -7|: 查找最近 7 天内访问过的文件。您可以根据需要调整数字以查找不同时间范围内的文件 
\subsubsection{检查文件是否存在}
FILE="/home/learn/file.txt"

if [ -f "\$FILE" ]; then 

     echo "\$FILE 存在。"
     
else 

     echo "\$FILE 不存在。"
     
fi  
\subsubsection{遍历文件}
for FILE in *; do

if [ -f "\$FILE" ]; then

echo "文件: \$FILE"

fi

done  
\subsubsection{计算1到10的总和}
sum=0 

for i in \{1..10\}; do

sum=\$((sum + i))

done 

echo "1 到 10 的总和是: \$sum"  
\subsubsection{简单输入输出}
if [ \$# -eq 0 ]; then

echo "请提供参数。"

exit 1

fi 

echo "您提供的参数是: \$1"   
\subsubsection{简单输出日期}
\#!/bin/bash 

echo "当前日期和时间: \$(date)" 
\subsection{Vim}
\subsubsection{打开文件}
vim learn.sh 
\subsubsection{插入模式}
按 \verb|i| 进入插入模式,可以开始编辑文本 
\subsubsection{保存文件 }
在普通模式下,输入 \verb|:w| 并按 \verb|Enter| 保存文件
\subsubsection{退出Vim}
输入 \verb|:q| 并按 \verb|Enter| 退出Vim。如果文件有修改,需先保存
\subsubsection{保存并退出}
输入 \verb|:wq| 或 \verb|:x| 并按 \verb|Enter| 保存文件并退出 
\subsubsection{强制退出}
输入 \verb|:q!| 并按 \verb|Enter| 强制退出,不保存修改 
\subsubsection{删除一行}
在普通模式下,输入 \verb|dd| 删除当前行 
\subsubsection{复制粘贴}
复制当前行:输入 \verb|yy|

粘贴:在光标位置输入 
\subsubsection{撤销和重做}
撤销上一步操作:输入 \verb|u| 

重做上一步操作:输入 \verb|Ctrl + r| 
\subsubsection{搜索文本}
输入 \verb|/search_term| 并按 \verb|Enter| 搜索指定的文本,使用 \verb|n| 跳转到下一个匹配项 
\section{使用感悟}
\subsection{Shell和脚本}
\subsubsection{提高效率}
通过编写脚本,可以将一些重复性的任务自动化,从而节省时间和精力。这对于我们处理大量数据或执行复杂操作的场景非常有用。
\subsubsection{调试困难}
由于shell脚本是逐行执行的,调试起来有难度。我需要仔细检查每一行代码,确保没有语法错误和逻辑错误。 
\subsection{Vim}
\subsubsection{模式切换}
vim拥有多个模式,包括命令模式、插入模式和可视模式。这需要我们熟练掌握每个模式的切换方法。
\subsubsection{快捷键}
vim拥有很多快捷键,这让我们在不用鼠标的情况下就可以完成文本编辑。熟练掌握这些快捷键可以提高我们文本编辑的速度。
\section{github地址}
\url{https://github.com/211-sss/homework}
\end{document}
