\documentclass{ctexart}
\usepackage{graphicx} % Required for inserting images
\usepackage{hyperref}
\title{Git实验报告}
\author{耿惠生 23020007028}
\date{September 2024}

\begin{document}

\maketitle

\section{练习内容}
\subsection{git}
\subsubsection{创建仓库,新建文件并提交}
git init

echo "first" > file.txt

git add file.txt

git commit -m"第一次提交"
\subsubsection{查看当前工作区状态}
git status
\subsubsection{查看提交历史}
git log

git log -- online
\subsubsection{向库中添加文件并提交,然后将它从历史中删除}
echo "First" > file.txt

git add file.txt

git commit -m"提交"

git filter-repo --path file.txt --invert-paths 

git push origin main --force

git log --all -- file.txt
\subsubsection{课后习题5}
nano \~/.gitconfig  

graph = log --all --graph --decorate --oneline 

git graph 
\subsubsection{创建全局忽略规则 }
git config --global core.excludesfile \~/.gitignore\_global 
\subsubsection{创建并查看分支}
git branch new-feature

git checkout new-feature 

git branch 
\subsubsection{合并并删除分支}
git checkout main

git merge new-feature 

git branch -d new-feature 
\subsubsection{远端操作}
git remote add origin https://missing-semester-cn.github.io/

git remote -v 

git clone 
\subsubsection{高级操作}
git add -p :交互式暂存
git rebase -i :交互式变基
git blame  :查看最后修改某行的人 
git stash  :暂时移除工作目录下的修改内容 
\subsection{Latex}
\subsubsection{文档类型}
documentclass\{类型\}
\subsubsection{标题、作者、日期}
\textbackslash{}title\{标题\}

\textbackslash{}author\{作者\}

\textbackslash{}date\{日期\}
\subsubsection{正文}
始于\textbackslash{}begin\{docement\}

终于\textbackslash{}end\{document\}
\subsubsection{显示前言区信息}
\textbackslash{} maketitle
\subsubsection{斜体、粗体、下划线}
\textbackslash{}textit
斜体

\textbackslash{}underline
下划线

\textbackslash{}textbf
粗体
\subsubsection{章节}
\textbackslash{}section
章节

\textbackslash{}subsection
子章节

\textbackslash{}subsubsection
三级章节
\subsubsection{添加图片}
\textbackslash{}includegraphics\{图片名称\}
\subsubsection{有序列表}
\textbackslash{}begin\{enumerate\}

\textbackslash{}item
列表项1

\textbackslash{}item
列表项2

\textbackslash{}end\{enumerate\}
\subsubsection{无序列表}
只需把有序列表中的"enumerate"改为"itemize"
\subsubsection{表格}
\textbackslash{}begin\{tabular\}\{c c c\} 
c c c 表示表格有三列

单元格 1 \& 单元格 2  \& 单元格 3

单元格 4 \& 单元格 5  \& 单元格 6

单元格 7 \& 单元格 8  \& 单元格 9

\textbackslash{}end\{tabular\}
\section{使用感悟}
\subsection{git}
\subsubsection{版本控制的重要性}
Git 允许我们跟踪项目的每一次更改,能够随时查看历史记录,了解项目的演变过程。 
\subsubsection{分支管理的灵活性}
通过创建分支,团队成员可以在不同的功能上并行工作,避免了相互干扰。 
\subsection{Latex}
LaTeX 自动处理文档的格式,确保整个文档的一致性,减少了手动调整的需求。 

LaTeX 强调文档的结构化,使用章节、节、子节等层次分明的组织方式,使得文档逻辑清晰。 
\section{github}
\url{https://github.com/211-sss/homework}
\end{document}